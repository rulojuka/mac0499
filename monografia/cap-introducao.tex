%% ------------------------------------------------------------------------- %%
\chapter{Introdução}
\label{cap:introducao}

\begin{itemize}
\item Explicar aqui qual é o problema. Primeiro começando com uma visão geral do que é a transcrição e como isso é normalmente feito por um músico. Apresentar a entrada e a saída esperada do programa e também apresentar por cima os critérios de eficiência (tempo real) e robustez.
\item Apresentar como motivação as várias áreas de computação musical que utilizam uma estimação de F0 ou transcrição melódica.
\item Dar vários exemplos de programas que já fazem transcrição ou mesmo estimação de F0. Afinadores de violão, Guitar Hero/Rock Band, Shazam, ScoreCloud, PitchScope, intelliScore Ensemble.
\item Mostrar como uma interface gráfica pode ajudar um músico.
\end{itemize}

%
%Para a escrita de textos em Ciência da Computação, o livro de Justin Zobel, 
%\emph{Writing for Computer Science} \citep{zobel:04} é uma leitura obrigatória. 
%O livro \emph{Metodologia de Pesquisa para Ciência da Computação} de 
%\citet{waz:09} também merece uma boa lida.

%O uso desnecessário de termos em lingua estrangeira deve ser evitado. No entanto,
%quando isso for necessário, os termos devem aparecer \emph{em itálico}.

%\begin{small}
%\begin{verbatim}
%Modos de citação:
%indesejável: [AF83] introduziu o algoritmo ótimo.
%indesejável: (Andrew e Foster, 1983) introduziram o algoritmo ótimo.
%certo : Andrew e Foster introduziram o algoritmo ótimo [AF83].
%certo : Andrew e Foster introduziram o algoritmo ótimo (Andrew e Foster, 1983).
%certo : Andrew e Foster (1983) introduziram o algoritmo ótimo.
%\end{verbatim}
%\end{small}

%Uma prática recomendável na escrita de textos é descrever as legendas das
%figuras e tabelas em forma auto-contida: as legendas devem ser razoavelmente
%completas, de modo que o leitor possa entender a figura sem ler o texto onde a
%figura ou tabela é citada.  

%Apresentar os resultados de forma simples, clara e completa é uma tarefa que
%requer inspiração. Nesse sentido, o livro de \citet{tufte01:visualDisplay},
%\emph{The Visual Display of Quantitative Information}, serve de ajuda na
%criação de figuras que permitam entender e interpretar dados/resultados de forma
%eficiente.

