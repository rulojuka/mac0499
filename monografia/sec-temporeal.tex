%% ------------------------------------------------------------------------- %%
\section{Transcrição em Tempo Real}
\label{sec:temporeal}

\begin{itemize}
\item Mostrar as necessidades mostradas no artigo (De La Cuadra, Patricio, Aaron Master, and Craig Sapp.``Efficient pitch detection techniques for interactive music.'')
	\begin{itemize}
	\item Resolução em frequência de pelo menos semi-tom, incluindo a oitava correta
	\item Reconhecimento em tempo hábil
	\item Utilização de instrumentos com harmônicos bem comportados
	\end{itemize}
\item Contrastar com necessidades sem tempo real, como detecção de início e fim de nota, por exemplo
\end{itemize}

Aqui vai o grosso da monografia: explicar detalhadamente e mostrar o código de cada uma das técnicas:

\begin{itemize}
\item Harmonic Product Spectrum
\item Maximum Likelihood
\item Cepstrum-Biased Harmonic Product Spectrum
\item Weighted Autocorrelation Function
\end{itemize}