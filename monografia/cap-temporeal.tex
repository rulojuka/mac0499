%% ------------------------------------------------------------------------- %%
\chapter{Transcri��o em Tempo Real}
\label{cap:temporeal}

\begin{itemize}
\item Mostrar as necessidades mostradas no artigo 
	\begin{itemize}
	\item Resolu��o em frequ�ncia de pelo menos semi-tom, incluindo a oitava correta
	\item Reconhecimento em tempo h�bil
	\item Utiliza��o de instrumentos com harm�nicos bem comportados
	\end{itemize}
\item Contrastar com necessidades sem tempo real, como detec��o de in�cio e fim de nota, por exemplo
\item Citar que o ASyMuT funciona bem r�pido mas n�o necessariamente em tempo real (fim de nota, processador lento, etc)
\end{itemize}

Aqui vai o grosso da monografia: explicar detalhadamente cada uma das t�cnicas:

\begin{itemize}
\item Harmonic Product Spectrum
\item Maximum Likelihood
\item Cepstrum-Biased Harmonic Product Spectrum
\item Weighted Autocorrelation Function
\end{itemize}